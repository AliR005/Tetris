\documentclass{article}
\usepackage[utf8]{inputenc}
\usepackage[T2A]{fontenc}
\usepackage[russian]{babel}

\title{Tetris in C. Documentation}
\author{AliR}
\begin{document}
\maketitle

\section{Введение}
Тетрис — это классическая аркадная игра, в которой игрок управляет падающими блоками (тетромино), с целью заполнения горизонтальных линий на игровом поле. Игра заканчивается, когда блоки достигают верхней части поля. Этот проект реализует игру Тетрис на языке программирования C с использованием библиотеки ncurses для управления графикой и вводом.

\section{Установка}
Для установки игры выполните следующие шаги:
\begin{enumerate}
    \item Соберите проект с помощью Makefile:
    \begin{verbatim}
    make
    \end{verbatim}
    \item Установите игру:
    \begin{verbatim}
    make install (Установка произойдет в "/usr/local/bin")
    Либо: make install DESTDIR="/ваш путь"
    \end{verbatim}
\end{enumerate}

\section{Использование}
После установки запустите игру, выполнив команду:
\begin{verbatim}
./Tetris
\end{verbatim}
Управление осуществляется с помощью клавиш:
\begin{itemize}
    \item Enter: начало игры.
    \item p: пауза.
    \item Esc: выход из игры.
    \item Key Left: перемещение блока влево.
    \item Key Right: перемещение блока вправо.
    \item Key Down: ускорение падения блока.
    \item Space: вращение блока.
\end{itemize}

\end{document}
